\documentclass{article}
\usepackage{amsmath, amssymb}
\usepackage{tabularx}
\usepackage{geometry}
\geometry{margin=1in}
\usepackage{float}
\usepackage{graphicx}

% --- Custom Command Definitions ---
\newcommand{\ruleofinference}[3]{
\begin{tabular}{rl}
     & \rule{0pt}{2em}$#1$ \\
     & $#2$ \\\cline{2-2}
     $\therefore$ & $#3$
\end{tabular}
}

\newcommand{\onelinerrule}[2]{
\begin{tabular}{rl}
     & \rule{0pt}{2em}$#1$ \\\cline{2-2}
     $\therefore$ & $#2$
\end{tabular}
}

\newcommand{\twolineequivalence}[2]{
\begin{tabular}{l}
     \rule{0pt}{2em}$#1$ \\
     $#2$
\end{tabular}
}

\newcommand{\onelineequivalence}[1]{
\begin{tabular}{l}
     \rule{0pt}{2em}$#1$
\end{tabular}
}
% ----------------------------------

\title{Definitions, Laws, and Axioms}
\author{Tanisha Gupta, Aidan Nguyen, Jennifer Jiang, Parth Parikh}
\date{Spring 2026}


\begin{document}
\maketitle

\noindent
This document contains the official list of definitions, laws, and axioms that students are permitted to reference and use for CS 2050 assignments and examinations.


\section{Proof Tips}

This section outlines some common mistakes and key issues we find in proofs. While not an exhaustive list, the items listed below will result in \textbf{definite point deductions} on homeworks, exams, and any other assignments for this course. We recommend you pay careful attention to this list as we will continue to update it frequently. 

\begin{itemize}
    \item \textbf{Proper Use of Closure:} When invoking the Axiom of Closure, you must specify both the operation (addition or multiplication) and the specific domain. Closure is officially accepted for multiplication and addition under the domains of $\mathbb{Z}$ and $\mathbb{R}$. \textbf{\textit{Note:}} Integers are NOT closed over division!
    
    \item \textbf{Avoid Assuming the Conclusion:} A proof must demonstrate that the conclusion follows from the premises. An argument is only valid if the truth of the premises guarantees the conclusion. Starting a proof by assuming the conclusion is true is a logical error; the final proof must move from premises to conclusion.
    
    \item \textbf{Parity Definitions:} Always use the formal definitions for even and odd integers. An integer $x$ is even if it can be represented as $x = 2a, a \in \mathbb{Z}$. An integer $x$ is odd if it can be represented as $x = 2a + 1, a \in \mathbb{Z}$. 
    
    \item \textbf{Scratch Work vs. Formal Proof:} While scratch work often involves working backwards from the conclusion to the premises or using a two-column format, the final proof should be presented as a clear, forward-moving logical progression in \textbf{paragraph format}.
    
    \item \textbf{Citing Domains for New Variables:} Whenever a new variable is introduced, you must explicitly cite its domain. This is critical because predicates, quantifiers, and general givens are typically only defined within a specified domain. 
    
    \item \textbf{Variable Domains in Sets:} Ensure you are using the correct notation for number sets when defining variables, particularly for the following:
    \begin{itemize}
        \item \( \mathbb{N} \): Natural numbers \(\{0, 1, 2, 3, \dots\}\)
        \item \( \mathbb{Z} \): Integers \(\{\dots, -2, -1, 0, 1, 2, \dots\}\)
        \item \( \mathbb{Q} \): Rational numbers \(\left\{ \frac{a}{b} \mid a, b \in \mathbb{Z}, b \neq 0 \right\}\)
        \item \( \mathbb{R} \): Real numbers
    \end{itemize}
\end{itemize}

\section{Definitions}

% \begin{enumerate}
%     \item \textbf{Definition of Even}: If a number $x$ is even, it can be expressed as $x=2k, k\in \mathbb{Z}$
% \end{enumerate}

\begin{enumerate}
    \item \textbf{Negation ($\lnot$): } the negation of $p$ is the statement “not $p$.”
    \item \textbf{Conjunction ($\land$):} the conjunction of $p$ and $q$ is the statement “$p$ and $q$.” For the statement to be true, both $p$ and $q$ must be true.
    \item \textbf{Disjunction ($\lor$):} the disjunction of $p$ and $q$ is the statement “$p$ or $q$.” For the statement to be true, at least one of $p$ or $q$ must be true.
    \item \textbf{Exclusive-or ($\oplus$):} the exclusive-or of $p$ and $q$ is the statement “$p$ XOR $q$.” For the statement to be true, exactly one of $p$ and $q$ must be true.
    \item \textbf{Conditional statement ($\rightarrow$):} “If $p$, then $q$.” Written $p \rightarrow q$. This is false when $p$ is true and $q$ is false, and true in every other instance. 
    \item \textbf{Converse:} $q \rightarrow p$.
    \item \textbf{Inverse:} $\lnot p \rightarrow \lnot q$.
    \item \textbf{Contrapositive:} $\lnot q \rightarrow \lnot p$ (logically equivalent to the given conditional).
    \item \textbf{Bi-conditional statement ($\leftrightarrow$):} “$p$ if and only if $q$.” One cannot exist without the other. Statement is true when both statements are true or both statements are false. $q$ is sufficient for $p$ and $p$ is sufficient for $q$. 
    \item \textbf{Definition of Biconditional:} $$p \leftrightarrow q \equiv (p \rightarrow q) \land (q \rightarrow p)$$
    \item \textbf{Tautology:} a compound proposition that is always true. Examples: $p \rightarrow p, p \lor \lnot p$.
    \item \textbf{Contradiction:} a compound proposition that is always false. Example: $\lnot p \land p$.
    \item \textbf{Contingency:} a compound proposition that is neither a tautology nor a contradiction. Most statements are contingencies. Example: $p \land q$.
    \item \textbf{Logical Equivalence:} Two compound propositions $p$ and $q$ are logically equivalent if $p \leftrightarrow q$ is a tautology. Notation: $p \equiv q$.
    \item \textbf{Predicate (Propositional Function):} A predicate is a statement involving one or more variables that becomes either true or false when values are assigned to the variables. Written like $P(x)$.
    \item \textbf{Universal Quantifier ($\forall$):} The universal quantifier $\forall$ is read as “for all” or “for every.” The statement $\forall x P(x)$ means that the predicate $P(x)$ is true for every element $x$ in the specified domain.
    \item \textbf{Existential Quantifier ($\exists$):} The existential quantifier $\exists$ is read as “there exists.” The statement $\exists x P(x)$ means that there is at least one element $x$ in the domain for which $P(x)$ is true.
    \item \textbf{Unique Existential Quantifier ($\exists !$):} The unique existential quantifier $\exists !$ is read as “there exists exactly one.” The statement $\exists ! x P(x)$ means that there is exactly one element $x$ in the domain for which the predicate $P(x)$ is true.
    \item \textbf{Sets of Numbers:}
        \begin{itemize}
            \item \( \mathbb{N} \): Natural numbers \(\{0, 1, 2, 3, \dots\}\) 
            \item \( \mathbb{Z} \): Integers \(\{\dots, -2, -1, 0, 1, 2, \dots\}\)
            \item \( \mathbb{Q} \): Rational numbers \(\left\{ \frac{a}{b} \mid a, b \in \mathbb{Z}, b \neq 0 \right\}\)
            \item \( \mathbb{R} \): Real numbers (all points on the number line)
            \item \( \mathbb{C} \): Complex numbers \(\{a + bi \mid a, b \in \mathbb{R}, i^2 = -1\}\)
            \item \( \mathbb{I} \): Irrational numbers (reals not in \(\mathbb{Q}\), e.g., \(\pi, \sqrt{2}\))
            \item \( \mathbb{Z}^+ \): Positive integers \(\{1, 2, 3, \dots\}\)
            \item \( \mathbb{Z}^- \): Negative integers \(\{-1, -2, -3, \dots\}\)
        \end{itemize}

    \item \textbf{Nested Quantifier: } When one quantifier is within the scope of another. Ex: $\forall x \exists y(x-2y=5)$
    \item \textbf{Argument: } Is a collection of premises and a conclusion. An argument is valid if and only if the truth of the premises guarantees the conclusion. Or, in the language of the class (where $t$ is the conclusion and $p_1, p_2, \cdots, p_k$ are premises): $$(p_1 \land p_2 \land \dots p_n) \to t$$

    \item \textbf{English Conditionals (the non-intuitive ones have been bolded):}

        \begin{itemize}
            \item If $p$, then $q$.
            \item $p$ is sufficient for $q$.
            \item If $p$, $q$.
            \item \textbf{q if p.}
            \item $q$ when $p$.
            \item A necessary condition for $p$ is $q$.
            \item $q$ unless $\lnot p$.
            \item $p$ implies $q$.
            \item \textbf{p only if q.}
            \item A sufficient condition for $q$ is $p$.
            \item $q$ whenever $p$.
            \item $q$ is necessary for $p$.
            \item $q$ follows from $p$.
            \item $q$ provided that $p$. 
        \end{itemize}

    \item \textbf{Even}: An integer is even if it can be represented as two times another integer, or in other words, an integer $x$ is even if $x = 2a, a \in \mathbb{Z}.$
    \item \textbf{Odd}: An integer is odd if it can be represented as two times another integer plus 1, or in other words, an integer $x$ is odd if $x = 2a +1, a \in \mathbb{Z}.$
    \item \textbf{Perfect Square}: An integer $x$ is a perfect square if it can be written as $x = a^2, a \in \mathbb{Z}.$
    \item \textbf{Prime: } A prime number $x$ is a natural number greater than 1 that has exactly two distinct factors: 1 and itself. 
    
    
\end{enumerate}
    



\section{Laws \& Axioms}

\begin{itemize}
    \item \textbf{De Morgan's Laws: } 
        \begin{enumerate}
            \item $\lnot(p \land q) \equiv \lnot p \lor \lnot q$
            \item $\lnot(p \lor q) \equiv \lnot p \land \lnot q$
        \end{enumerate}
    \item \textbf{Conditional Disjunction Equivalence (CDE):} $p \rightarrow q \equiv \lnot p \lor q$.
    
\end{itemize}


\textbf{Axioms:}
\begin{itemize}
    \item Closure: Multiplication and Addition are closed under the domains of $\mathbb{Z}, \mathbb{Q}, \text{and}, \mathbb{R}.$
\end{itemize}


\newpage
% \newcommand{\twolineequivalence}[2]{
% \begin{tabular}{l}
%      \rule{0pt}{2em}$#1$ \\
%      $#2$
% \end{tabular}
% }

% \newcommand{\onelineequivalence}[1]{
% \begin{tabular}{l}
%      \rule{0pt}{2em}$#1$
% \end{tabular}
% }

\begin{center}
    \textbf{Only equivalences and rules of inferences present on these tables will be accepted}
\end{center}

\begin{center}
\begin{tabular}{p{0.4\textwidth}p{0.4\textwidth}}
    \textbf{\textit{Equivalence}} & \textbf{\textit{Name}}\\ 
    \hline
    \twolineequivalence{p \rightarrow q \equiv \neg p \lor q}{} & Conditional Disjunction Equivalence \\
    \twolineequivalence{p \rightarrow q \equiv \neg q \rightarrow \neg p}{} & Contrapositive Law \\
    \twolineequivalence{p \leftrightarrow q \equiv (p \rightarrow q) \land (q \rightarrow p )}{} & Definition of Biconditional \\
    \twolineequivalence{\lnot (\lnot p) \equiv p}{} & Double Negation law \\
    \twolineequivalence{p \land \textbf{T} \equiv p}{p \lor \textbf{F} \equiv p} & Identity Laws \\
    \twolineequivalence{p \lor \textbf{T} \equiv \textbf{T}}{p \land \textbf{F} \equiv \textbf{F}} & Domination Laws \\
    \twolineequivalence{p \lor p \equiv p}{p \land p \equiv p} & Idempotent Laws \\
    \twolineequivalence{\lnot (\lnot p) \equiv p}{} & Double Negation law \\
    \twolineequivalence{p \lor q \equiv q \lor p}{p \land q \equiv q \land p} & Commutative Laws \\
    \twolineequivalence{(p \lor q) \lor r \equiv p \lor (q \lor r)}{(p \land q) \land r \equiv p \land (q \land r)} & Associative Laws \\
    \twolineequivalence{p \lor (q \land r) \equiv (p \lor q) \land (p \lor r)}{p \land (q \lor r) \equiv (p \land q) \lor (p \land r)} & Distributive Laws \\
    \twolineequivalence{\neg (p \land q) \equiv \neg p \lor \neg q}{\neg (p \lor q) \equiv \neg p \land \neg q} & DeMorgan's Laws \\
    \twolineequivalence{p \lor (p \land q) \equiv p}{p \land (p \lor q) \equiv p} & Absorption Laws \\
    \twolineequivalence{p \lor \lnot p \equiv \textbf{T}}{p \land \lnot p \equiv \textbf{F}} & Negation Laws \\
\end{tabular}
\end{center}
\newpage
% \newcommand{\ruleofinference}[3]{
% \begin{tabular}{rl}
%      & \rule{0pt}{2em}$#1$ \\
%      & $#2$ \\\cline{2-2}
%      $\therefore$ & $#3$
% \end{tabular}
% }

% \newcommand{\onelinerrule}[2]{
% \begin{tabular}{rl}
%      & \rule{0pt}{2em}$#1$ \\\cline{2-2}
%      $\therefore$ & $#2$
% \end{tabular}
% }


\textbf{Only equivalences and rules of inferences present on these tables will be accepted}

\begin{center}
\begin{tabular}{p{0.4\textwidth}p{0.4\textwidth}}
    \textbf{\textit{Rule of Inference}} & \textbf{\textit{Name}}\\ \hline
    \ruleofinference{p}{p \rightarrow q}{q} & Modus ponens \\
    \ruleofinference{\lnot q}{p \rightarrow q}{\lnot p} & Modus tollens \\
    \ruleofinference{p \rightarrow q}{q \rightarrow r}{p \rightarrow r} & Hypothetical syllogism \\
    \ruleofinference{p \lor q}{\lnot p}{q} & Disjunctive syllogism \\
    \onelinerrule{p}{p \lor q} & Addition \\
    \onelinerrule{p \land q}{p} & Simplification \\
    \ruleofinference{p}{q}{p \land q} & Conjunction \\
    \ruleofinference{p \lor q}{\lnot p \lor r}{q \lor r} & Resolution
\end{tabular}
\end{center}
\begin{center}
\begin{tabular}{p{0.4\textwidth}p{0.4\textwidth}}
    \textbf{\textit{Rule of Inference}} & \textbf{\textit{Name}}\\ \hline
    \onelinerrule{\forall x P(x)}{P(c)} & Universal instantiation \\
    \onelinerrule{P(c) \text{ for an arbitrary } c}{\forall x P(x)} & Universal generalization \\
    \onelinerrule{\exists x P(x)}{P(c) \text{ for some element } c} & Existential instantiation \\
    \onelinerrule{P(c) \text{ for some element } c}{\exists x P(x)} & Existential generalization \\ \\
    \textbf{\textit{Equivalence}} & \textbf{\textit{Name}}\\ \hline
    \twolineequivalence{\neg \exists P(x)\equiv \forall \neg P(x)}{\neg \forall P(x)\equiv \exists \neg P(x)} & DeMorgan's Laws for Quantifiers
\end{tabular}
\end{center}

\end{document}
