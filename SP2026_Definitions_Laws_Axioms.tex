\documentclass{article}
\usepackage{amsmath, amssymb}
\usepackage{tabularx}
\usepackage{geometry}
\geometry{margin=1in}
\usepackage{float}
\usepackage{graphicx} % Required for inserting images

\title{Definitions, Laws, and Axioms}
\author{Tanisha Gupta, Aidan Nguyen, Jennifer Jiang, Parth Parikh}
\date{Spring 2026}


% Define helper command for two-line equivalences
\newcommand{\twolineequivalence}[2]{
\begin{tabular}{l}
     \rule{0pt}{2em}$#1$ \\
     $#2$
\end{tabular}
}

\begin{document}
\maketitle

\noindent
This document contains the official list of definitions, laws, and axioms that students are permitted to reference and use for CS 2050 assignments and examinations.

\section{Definitions}

% \begin{enumerate}
%     \item \textbf{Definition of Even}: If a number $x$ is even, it can be expressed as $x=2k, k\in \mathbb{Z}$
% \end{enumerate}

\begin{enumerate}
    \item \textbf{Negation ($\lnot$): } the negation of $p$ is the statement “not $p$.”
    \item \textbf{Conjunction ($\land$):} the conjunction of $p$ and $q$ is the statement “$p$ and $q$.” For the statement to be true, both $p$ and $q$ must be true.
    \item \textbf{Disjunction ($\lor$):} the disjunction of $p$ and $q$ is the statement “$p$ or $q$.” For the statement to be true, at least one of $p$ or $q$ must be true.
    \item \textbf{Exclusive-or ($\oplus$):} the exclusive-or of $p$ and $q$ is the statement “$p$ XOR $q$.” For the statement to be true, exactly one of $p$ and $q$ must be true.
    \item \textbf{Conditional statement ($\rightarrow$):} “If $p$, then $q$.” Written $p \rightarrow q$. This is false when $p$ is true and $q$ is false, and true in every other instance. 
    \item \textbf{Converse:} $q \rightarrow p$.
    \item \textbf{Inverse:} $\lnot p \rightarrow \lnot q$.
    \item \textbf{Contrapositive:} $\lnot q \rightarrow \lnot p$ (logically equivalent to the given conditional).
    \item \textbf{Bi-conditional statement ($\leftrightarrow$):} “$p$ if and only if $q$.” One cannot exist without the other. Statement is true when both statements are true or both statements are false. $q$ is sufficient for $p$ and $p$ is sufficient for $q$. 
    \item \textbf{Definition of Biconditional:} $$p \leftrightarrow q \equiv (p \rightarrow q) \land (q \rightarrow p)$$
    \item \textbf{Tautology:} a compound proposition that is always true. Examples: $p \rightarrow p, p \lor \lnot p$.
    \item \textbf{Contradiction:} a compound proposition that is always false. Example: $\lnot p \land p$.
    \item \textbf{Contingency:} a compound proposition that is neither a tautology nor a contradiction. Most statements are contingencies. Example: $p \land q$.
    \item \textbf{Logical Equivalence:} Two compound propositions $p$ and $q$ are logically equivalent if $p \leftrightarrow q$ is a tautology. Notation: $p \equiv q$.
    \item \textbf{Predicate (Propositional Function):} A predicate is a statement involving one or more variables that becomes either true or false when values are assigned to the variables. Written like $P(x)$.
    \item \textbf{Universal Quantifier ($\forall$):} The universal quantifier $\forall$ is read as “for all” or “for every.” The statement $\forall x P(x)$ means that the predicate $P(x)$ is true for every element $x$ in the specified domain.
    \item \textbf{Existential Quantifier ($\exists$):} The existential quantifier $\exists$ is read as “there exists.” The statement $\exists x P(x)$ means that there is at least one element $x$ in the domain for which $P(x)$ is true.
    \item \textbf{Unique Existential Quantifier ($\exists !$):} The unique existential quantifier $\exists !$ is read as “there exists exactly one.” The statement $\exists ! x P(x)$ means that there is exactly one element $x$ in the domain for which the predicate $P(x)$ is true.
    \item \textbf{Sets of Numbers:}
        \begin{itemize}
            \item \( \mathbb{N} \): Natural numbers \(\{0, 1, 2, 3, \dots\}\) 
            \item \( \mathbb{Z} \): Integers \(\{\dots, -2, -1, 0, 1, 2, \dots\}\)
            \item \( \mathbb{Q} \): Rational numbers \(\left\{ \frac{a}{b} \mid a, b \in \mathbb{Z}, b \neq 0 \right\}\)
            \item \( \mathbb{R} \): Real numbers (all points on the number line)
            \item \( \mathbb{C} \): Complex numbers \(\{a + bi \mid a, b \in \mathbb{R}, i^2 = -1\}\)
            \item \( \mathbb{I} \): Irrational numbers (reals not in \(\mathbb{Q}\), e.g., \(\pi, \sqrt{2}\))
            \item \( \mathbb{Z}^+ \): Positive integers \(\{1, 2, 3, \dots\}\)
            \item \( \mathbb{Z}^- \): Negative integers \(\{-1, -2, -3, \dots\}\)
        \end{itemize}
    
    
\end{enumerate}
    



\section{Laws \& Axioms}

\begin{itemize}
    \item \textbf{De Morgan's Laws: } 
        \begin{enumerate}
            \item $\lnot(p \land q) \equiv \lnot p \lor \lnot q$
            \item $\lnot(p \lor q) \equiv \lnot p \land \lnot q$
        \end{enumerate}
    \item \textbf{Conditional Disjunction Equivalence (CDE):} $p \rightarrow q \equiv \lnot p \lor q$.
\end{itemize}

\begin{table}[H]
\renewcommand{\arraystretch}{1.3} % increase row height
\centering
\caption{\textbf{Only equivalences and rules of inference present on this table will be accepted.}}
\begin{tabularx}{\textwidth}{>{\raggedright\arraybackslash}X >{\raggedright\arraybackslash}X}
    \textbf{\textit{Equivalence}} & \textbf{\textit{Name}}\\ 
    \hline
    \twolineequivalence{p \rightarrow q \equiv \neg p \lor q}{} & Conditional Disjunction Equivalence \\
    \twolineequivalence{p \rightarrow q \equiv \neg q \rightarrow \neg p}{} & Contrapositive Law \\
    \twolineequivalence{p \leftrightarrow q \equiv (p \rightarrow q) \land (q \rightarrow p)}{} & Definition of Biconditional \\
    \twolineequivalence{\lnot (\lnot p) \equiv p}{} & Double Negation Law \\
    \twolineequivalence{p \land \textbf{T} \equiv p}{p \lor \textbf{F} \equiv p} & Identity Laws \\
    \twolineequivalence{p \lor \textbf{T} \equiv \textbf{T}}{p \land \textbf{F} \equiv \textbf{F}} & Domination Laws \\
    \twolineequivalence{p \lor p \equiv p}{p \land p \equiv p} & Idempotent Laws \\
    \twolineequivalence{p \lor q \equiv q \lor p}{p \land q \equiv q \land p} & Commutative Laws \\
    \twolineequivalence{(p \lor q) \lor r \equiv p \lor (q \lor r)}{(p \land q) \land r \equiv p \land (q \land r)} & Associative Laws \\
    \twolineequivalence{p \lor (q \land r) \equiv (p \lor q) \land (p \lor r)}{p \land (q \lor r) \equiv (p \land q) \lor (p \land r)} & Distributive Laws \\
    \twolineequivalence{\neg (p \land q) \equiv \neg p \lor \neg q}{\neg (p \lor q) \equiv \neg p \land \neg q} & De Morgan's Laws \\
    \twolineequivalence{p \lor (p \land q) \equiv p}{p \land (p \lor q) \equiv p} & Absorption Laws \\
    \twolineequivalence{p \lor \lnot p \equiv \textbf{T}}{p \land \lnot p \equiv \textbf{F}} & Negation Laws \\
\end{tabularx}
\end{table}




\end{document}
